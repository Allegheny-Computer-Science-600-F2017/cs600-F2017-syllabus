\documentclass[11pt]{article}

% Use this when displaying a new command

\newcommand{\command}[1]{``\lstinline{#1}''}
\newcommand{\program}[1]{\lstinline{#1}}
\newcommand{\url}[1]{\lstinline{#1}}
\newcommand{\channel}[1]{\lstinline{#1}}
\newcommand{\option}[1]{``{#1}''}
\newcommand{\step}[1]{``{#1}''}

\usepackage{longtable}

\usepackage{pifont}
\newcommand{\checkmark}{\ding{51}}
\newcommand{\naughtmark}{\ding{55}}

\usepackage{listings}
\lstset{
  basicstyle=\small\ttfamily,
  columns=flexible,
  breaklines=true
}

% Define the headers and footers

\usepackage{fancyhdr}

\usepackage[margin=1in]{geometry}
\usepackage{fancyhdr}

\pagestyle{fancy}

\fancyhf{}
\rhead{Computer Science 610}
\lhead{Syllabus}
\rfoot{Page \thepage}
\lfoot{Spring 2018}

% Use elastic spacing around the headers

\usepackage{titlesec}
\titlespacing\section{0pt}{6pt plus 4pt minus 2pt}{4pt plus 2pt minus 2pt}

\newcommand{\syllabustitle}[1]
{
  \begin{center}
    \begin{center}
      \bf
      CMPSC 610\\Senior Thesis II\\
      Spring 2018\\
      \medskip
    \end{center}
    \bf
    #1
  \end{center}
}

\begin{document}

\thispagestyle{empty}

\syllabustitle{Syllabus}

\vspace*{-.2in}
\subsection*{Course Instructors}

\begin{tabular}{c c}

\begin{minipage}{3.5in}
Dr.\ Oliver Bonham-Carter \\
\noindent Office Location: Alden Hall 104 \\
\noindent Email: \url{obonhamcarter@allegheny.edu} \\
\end{minipage} &

\begin{minipage}{3.5in}
Dr.\ Janyl Jumadinova\\
\noindent Office Location: Alden Hall 105 \\
\noindent Email: \url{jjumadinova@allegheny.edu} \\
\end{minipage} \\

\begin{minipage}{3.5in}
Dr.\ Gregory M.\ Kapfhammer\\
\noindent Office Location: Alden Hall 108 \\
\noindent Email: \url{gkapfham@allegheny.edu} \\
\end{minipage} &

\begin{minipage}{3.5in}
Dr.\ Aravind Mohan\\
\noindent Office Location: Alden Hall Room 106\\
\noindent Email: \url{amohan@allegheny.edu} \\
\end{minipage}

\end{tabular}
\vspace*{-.2in}

\subsection*{Instructors' Office Hours}

Please visit the web sites of the course instructors to view their office hours. Using the ``appointment slots''
feature of Google Calendar, you can select an available meeting time. After picking your time slot, the reserved meeting
will appear in both your Google Calendar and the instructor's.

\vspace*{-.1in}
\begin{itemize}
  \itemsep -.25em
  \item Oliver Bonham-Carter: \url{http://www.cs.allegheny.edu/sites/obonhamcarter/}
  \item Janyl Jumadinova: \url{http://www.cs.allegheny.edu/sites/jjumadinova/}
  \item Gregory M.\ Kapfhammer: \url{http://www.cs.allegheny.edu/sites/gkapfham/}
  \item Aravind Mohan \url{http://www.cs.allegheny.edu/sites/amohan/}
\end{itemize}

\vspace*{-.25in}
\subsection*{Course Communication}

Throughout this semester, students and faculty will use last semester's Slack
team to support course communication. Whenever possible, students are also
encouraged to post appropriate questions to a channel in Slack, which is
available at \url{https://CMPSC600Fall2017.slack.com}. All students are also
required to use GitHub to submit all of the deliverables for this course's
various projects.

\vspace*{-.1in}
\subsection*{Course Schedule}

Organized according to the calendar month during which an activity takes place
or a project is due, the following table outlines this course's schedule for
the entire academic semester. Some of these dates are approximate and, if the
need to do so presents itself, it is possible for the course instructors to
modify the proposed schedule and notify the class of any changes via email or
Slack. Unless it is otherwise noted that there is no class session, it is
assumed that, even if there is a course project due or a research task to
complete, you will still attend a research group meeting during the scheduled
session for this course and participate in the technology tea.

\begin{center}
\begin{tabular}{r|l}
\hline

Before January 23  & Schedule weekly meeting time with your first reader \\
Before February 20 & Schedule thesis defense with Pauline Lanzine \\
March 6           & Submit three-paragraph status update on your progress \\
April 3            & Submit unbound digital copy of thesis to first and second readers by 4 pm \\
April 4--April 24  & Complete oral defense of senior thesis \\
May 1              & Submit final bound and signed thesis to Pauline Lanzine by 4 pm\\

\hline
Entire academic semester & Meet with first reader on a weekly basis \\
Entire academic semester & Participate in discussions in the Slack team \\
Entire academic semester & Use GitHub to commit to your research repositories \\
\hline
\end{tabular}
\end{center}

% \vspace*{-.25in}

\noindent Please note that, unless evidence of extenuating circumstances is
presented in writing to all of the instructors, a student's grade in the course
will be reduced if the stated deadlines are not met. Students who have questions
or concerns about these deadlines should talk with their first reader.

\subsection*{Required Textbooks}

\noindent{\em On Being a Scientist: A Guide to Responsible Conduct in Research}
(Third Edition).  Committee on Science, Engineering, and Public Policy, National
Academy of Sciences, National Academy of Engineering, and Institute of Medicine.
ISBN: 0309119715, 82 pages, 2009.\\ (References to the textbook are abbreviated
as ``OBAS'').

\vspace*{.1in}

\noindent{\em BUGS in Writing: A Guide to Debugging Your Prose} (Second
Edition). Lyn Dupr\'e.  Addison-Wesley Professional.  ISBN-10: 020137921X and
ISBN-13: 978-0201379211, 704 pages, 1998.\\ (References to the textbook are
abbreviated as ``BIW'').

\vspace*{.1in}

\noindent{\em Writing for Computer Science} (Second Edition).  Justin Zobel.
Springer ISBN-10: 1852338024 and ISBN-13: 978-1852338022, 270 pages, 2004. \\
(References to the textbook are abbreviated as ``WFCS'').

\subsection*{Overview of the Grading Policies}

Final grades are determined after the entire faculty of the Department of
Computer Science --- not just your course instructor for CMPSC 610 --- review
and discuss all of the submitted deliverables.

Your grade in CMPSC 610 will be based on a combination of the following
activities and deliverables. Percentages are not given because we recognize that
the senior thesis experience differs from one student to the next and that there
are many variables, such as the nature of the project and the availability of
external resources, that can influence the relative importance of these
criteria. However, it is important to note that a large percentage of your grade
depends upon your written thesis proposal, the oral defense of your thesis
proposal, and your two thesis chapters.

\begin{itemize}

  \item {\bf Class Participation}: This includes meeting regularly with your
    first reader. Although the exact details about frequency and length of each
    meeting must be established with your first reader, you should adhere to the
    previously stated schedule. Additionally, this also requires regular
    contributions, in the form of questions and comments, to the course's Slack
    team. As previously mentioned in the ``Course Schedule'' section, all
    students are required to attend all of the Tuesday class sessions and to
    fully participate in their research group meetings.

  \item {\bf Course Repositories}: This involves students creating, at minimum,
    a GitHub version control repository for each of the assigned course
    projects. Students should click the relevant link in the Slack team to
    accept and begin working on the assignment. Now, you may follow the
    instructions in your repository's README file to complete and submit the
    assignment, regularly using the Travis system and GitHub's tagging mechanism
    to release PDFs of your proposal with versions that adhere to the semantic
    versioning standard. Course instructors will only grade and provide feedback
    on projects that are stored and released through GitHub.

  \item {\bf Status Update}: This document should describe the progress that a
    student has made on completing the research for their senior thesis and
    completing the preliminary research needed to demonstrate its feasibility.
    Written with feedback from your first and second readers, your status update
    should be stored and released through the appropriate GitHub repository.

  \item {\bf Written Thesis}: In consultation with your first reader and in
    accordance with the stated deadlines, you must work out a schedule for
    completion of your thesis research and your written document. All senior
    theses are due, properly formatted and signed (but not bound), on the
    stated due date.  Working closely with your first reader, you must produce
    a thesis that both follows the department's style and adheres to
    professional standards of writing. Your grade in CMPSC 610 will be reduced
    if you fail to submit your unbound thesis on time.

    Following your defense, you must submit the bound copy of your senior
    thesis by the aforementioned due date.  This document must incorporate any
    changes that were requested by your first and second reader. Seniors who
    have not delivered the signed and bound copies of their senior thesis by
    the stated deadline will receive an incomplete and will not graduate.

  \item{\bf Thesis Defense}: The standards for this presentation are the same
    as for the proposal defense --- you must give a ten to fifteen minute
    presentation supported by polished slides and adhere to all of the other
    stated requirements for this deliverable. Part of your grade for this
    defense will depend on how well you are able to discuss aspects of your
    thesis, including implications of your work, connections between your
    research and other areas of computer science, and possible extensions or
    improvements of your research ideas. You are expected to work with your
    first reader in preparing your oral defense. Your grade in CMPSC 610 will
    be reduced if you do not schedule or conduct your thesis defense by the
    stated deadlines.

    Unless there are severe extenuating circumstances, students are not allowed
    to reschedule their defense once they have a confirmed date from Pauline
    Lanzine. To schedule your thesis defense, please check the Google Calendar
    of your first and second readers and come to Pauline's office with three
    dates and times that fit into your schedule and the schedules of your
    readers. Do not suggest dates and times that conflict with the schedules of
    your readers!

\end{itemize}

\subsection*{Additional Details About Course Expectations and Deliverables}

\subsubsection*{Class Participation}

Once your readers have been assigned, you must regularly meet with your first
reader, who will report on your participation when the department's faculty
meet to assign final grades for this course. Students are expected to come to
each meeting with a status update on their progress and a meeting agenda.
Students should conclude each meeting by listing the tasks that they want to
complete before the next meeting. In addition, students should regularly
participate in the discussions on the relevant channels in the Slack team for
our course. Your participation on Slack may involve giving a quick status
update to your first reader, inviting your first reader to examine a draft of
your proposal or compile and run a new version of a program, or, within the
bounds of the Honor Code, answering a question from another senior conducting
their thesis research.

Finally, all students are required to attend and actively participate in all of
the class sessions that will involve both meetings with their research group
and informal discussion during the departmental technology tea session. During
these meetings, please be prepared to regularly share status updates on your
progress towards completing your research. Evidence of regular participation,
in the format of either oral or written reports, must be submitted to the
course instructor; failure to participate and submit this evidence will result
in the reduction of your grade in CMPSC 610.

\subsubsection*{GitHub Repositories}

Every student must accept each of the course projects, thus creating a GitHub
repository customized for the student and that specific project. All of these
GitHub repositories should have a README file that clearly explains the steps
that a student took to complete and release the final version of the
assignment. In addition to containing the \LaTeX{} source code that fulfills
the assignment, each GitHub repository should feature releases of the compiled
PDF files that are tagged with numbers that adhere to the semantic versioning
standard described at \url{http://semver.org/}.

The release of a compiled PDF file can be accomplished automatically by using
both the tagging and releases feature provided by GitHub and, additionally, the
continuous integration system provided by Travis. Your first and second readers
will download, read, and comment on a released PDF at semantic version 1.0.0 or
higher. Students who are not able to automatically release PDFs of their
projects may instead manually create them by using the GitHub interface. Please
see an instructor if you have questions about using GitHub. Failure to either
regularly commit to your GitHub repositories or to make releases of your PDFs
will lead to a decrease in your final grade for CMPSC 610. Please note that we
will continue to use the last GitHub repository from CMPSC 600 to store the
final chapters of your thesis. Please see an instructor if you cannot use
GitHub.

\subsubsection*{Using Email}

Although we will primarily use Slack for class communication, we will sometimes
use email to send announcements about important class matters. It is your
responsibility to check your email at least once a day and to ensure that you
can reliably send and receive emails. This class policy is based on the
statement about the use of email that appears in {\em The Compass}, the student
handbook.

\subsection*{Honor Code}

The Academic Honor Program that governs the academic program at Allegheny
College is described in the Allegheny Academic Bulletin. The Honor Program
applies to all work that is submitted for academic credit or to meet non-credit
requirements for graduation at Allegheny College. This includes all work
assigned for these classes (e.g., source code, technical diagrams, and your
written content); deliverables that are nearly identical the work of others will
be taken as evidence of violating the Honor Code. All students who have enrolled
in the College will work under the Honor Program. Each student who has
matriculated at the College has acknowledged the following pledge:

\begin{quote}
I hereby recognize and pledge to fulfill my responsibilities, as defined in the
Honor Code, and to maintain the integrity of both myself and the College
community as a whole.
\end{quote}

\subsection*{Disability Services}

The Americans with Disabilities Act (ADA) is a federal anti-discrimination
statute that provides comprehensive civil rights protection for persons with
disabilities. Among other things, this legislation requires all students with
disabilities be guaranteed a learning environment that provides for reasonable
accommodation of their disabilities. Students with disabilities who believe they
may need accommodations in this class are encouraged to contact Disability
Services at 332--2898. Disability Services is part of the Learning Commons and
is located in Pelletier Library. Please do this as soon as possible to ensure
that approved accommodations are implemented in a timely fashion.

\subsection*{Welcome to an Adventure in Computer Science}

CMPSC 610 affords you the opportunity to pursue independent research in computer
science and to ensure that your work has a positive influence on your future
plans, the students and faculty at Allegheny College, and a broader society that
relies heavily on computer hardware and software. At the start of your senior
year, we invite you to pursue this class with great enthusiasm and vigor.

\end{document}
