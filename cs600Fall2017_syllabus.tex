\documentclass[11pt]{article}

% Use this when displaying a new command

\newcommand{\command}[1]{``\lstinline{#1}''}
\newcommand{\program}[1]{\lstinline{#1}}
\newcommand{\url}[1]{\lstinline{#1}}
\newcommand{\channel}[1]{\lstinline{#1}}
\newcommand{\option}[1]{``{#1}''}
\newcommand{\step}[1]{``{#1}''}

\usepackage{longtable}

\usepackage{pifont}
\newcommand{\checkmark}{\ding{51}}
\newcommand{\naughtmark}{\ding{55}}

\usepackage{listings}
\lstset{
  basicstyle=\small\ttfamily,
  columns=flexible,
  breaklines=true
}

% Define the headers and footers

\usepackage{fancyhdr}

\usepackage[margin=1in]{geometry}
\usepackage{fancyhdr}

\pagestyle{fancy}

\fancyhf{}
\rhead{Computer Science 600}
\lhead{Syllabus}
\rfoot{Page \thepage}
\lfoot{Fall 2017}

% Use elastic spacing around the headers

\usepackage{titlesec}
\titlespacing\section{0pt}{6pt plus 4pt minus 2pt}{4pt plus 2pt minus 2pt}

\newcommand{\syllabustitle}[1]
{
  \begin{center}
    \begin{center}
      \bf
      CMPSC 600\\Senior Thesis I\\
      Fall 2017\\
      \medskip
    \end{center}
    \bf
    #1
  \end{center}
}

\begin{document}

\thispagestyle{empty}

\syllabustitle{Syllabus}

\vspace*{-.3in}
\subsection*{Course Instructors}

\begin{tabular}{c c}

\begin{minipage}{3.5in}
Dr.\ Oliver Bonham-Carter \\
\noindent Office Location: Alden Hall 104 \\
\noindent Email: \url{obonhamcarter@allegheny.edu} \\
\end{minipage} &

\begin{minipage}{3.5in}
Dr.\ Janyl Jumadinova\\
\noindent Office Location: Alden Hall 105 \\
\noindent Email: \url{jjumadinova@allegheny.edu} \\
\end{minipage} \\

\begin{minipage}{3.5in}
Dr.\ Gregory M.\ Kapfhammer\\
\noindent Office Location: Alden Hall 108 \\
\noindent Email: \url{gkapfham@allegheny.edu} \\
\end{minipage} &

\begin{minipage}{3.5in}
Dr.\ Aravind Mohan\\
\noindent Office Location: Alden Hall Room 106\\
\noindent Email: \url{amohan@allegheny.edu} \\
\end{minipage}

\end{tabular}
\vspace*{-.3in}

\subsection*{Instructors' Office Hours}

Please visit the web sites of the course instructors to view their office hours. Using the ``appointment slots''
feature of Google Calendar, you can select an available meeting time. After picking your time slot, the reserved meeting
will appear in both your Google Calendar and the instructor's.

\vspace*{-.1in}
\begin{itemize}
  \itemsep -.25em
  \item Oliver Bonham-Carter: \url{http://www.cs.allegheny.edu/sites/obonhamcarter/}
  \item Janyl Jumadinova: \url{http://www.cs.allegheny.edu/sites/jjumadinova/}
  \item Gregory M.\ Kapfhammer: \url{http://www.cs.allegheny.edu/sites/gkapfham/}
  \item Aravind Mohan \url{http://www.cs.allegheny.edu/sites/amohan/}
\end{itemize}

\vspace*{-.25in}
\subsection*{Course Communication}

Throughout the semester, students and faculty will use Slack to support course
communication. Whenever possible, students are also encouraged to post
appropriate questions to a channel in Slack, which is available at
\url{https://CMPSC600Fall2017.slack.com}. Moreover, all students are required
to use GitHub repositories to submit all of the deliverables for this course's
various projects.

\vspace*{-.1in}
\subsection*{Course Schedule}

Organized according to the calendar month during which an activity takes place
or a project is due, the following table outlines this course's schedule for the
entire academic semester. Some of these dates are approximate and, if the need
to do so presents itself, it is possible for the course instructors to change
the proposed schedule and notify the class of changes via email or Slack. Unless
it is otherwise noted that there is no class session (e.g., on October 24, which
is Gator Day), it is assumed that you will attend a research group meeting
during a schedule class session.

\begin{center}
\begin{longtable}{r|l}

\hline

August 29 & No class on the first day of the semester \\

\hline

September 5  & Release Project One: Title, Abstract, Readers       \\
September 7  & Submit Project One: Title, Abstract, Readers        \\
September 8  & First and Second Readers Assigned                   \\
September 11 & Register for CMPSC 600 with First Reader            \\
September 12 & Release Project Two: Senior Thesis Proposal         \\
September 12 & Schedule Weekly Meeting Time With Your First Reader \\
September 19 & Release Project Three: Status Update \\
September 19 & Research Group Meeting \\
September 26 & Research Group Meeting \\

\hline

October 3 & Submit Project Three: Status Update \\
October 10 & Fall Break --- No Class \\
October 17 & Research Group Meeting\\
October 24 & Gator Day --- No Class \\
October 31 & Submit Project Two: Senior Thesis Proposal \\

\hline

November 7 & Release Project Three: Senior Thesis Chapters \\
November 7 & Schedule Proposal Defense with Pauline Lanzine \\
November 14 & Research Group Meeting \\
November 21 & Research Group Meeting \\
November 28 & Get Technical Report Number from Pauline Lanzine \\

\hline

December 5  & Research Group Meeting \\
December 12 & Submit Project Three: Senior Thesis Chapters \\

\hline

November 1 -- November 30 & Oral Defense of Thesis Proposal \\
November 6 -- November 10 & Register for CMPSC 610 with first reader             \\

\hline

September through December & Meet with first reader on a regular basis          \\
September through December & Communicate with instructors and students in Slack \\

\hline

\end{longtable}
\end{center}

\noindent Please note that, unless evidence of extenuating circumstances is
presented in writing to all of the course instructors, a student's grade in the
course will be reduced if the stated deadlines are not met. Students who have
questions or concerns about these deadlines should talk with their first
reader.

\vspace{-.15in}
\subsection*{Required Textbooks}
\vspace{-.05in}

\noindent{\em On Being a Scientist: A Guide to Responsible Conduct in Research} (Third Edition).  Committee on Science,
Engineering, and Public Policy, National Academy of Sciences, National Academy of Engineering, and Institute of
Medicine. ISBN: 0309119715, 82 pages, 2009.\\ (References to the textbook are abbreviated as ``OBAS'').

\noindent{\em BUGS in Writing: A Guide to Debugging Your Prose} (Second Edition). Lyn Dupr\'e.  Addison-Wesley
Professional.  ISBN-10: 020137921X and ISBN-13: 978-0201379211, 704 pages, 1998.\\ (References to the textbook are
abbreviated as ``BIW'').

\noindent{\em Writing for Computer Science} (Second Edition).  Justin Zobel.  Springer ISBN-10: 1852338024 and ISBN-13:
978-1852338022, 270 pages, 2004. \\ (References to the textbook are abbreviated as ``WFCS'').

\vspace*{-.15in}
\subsection*{Overview of the Grading Policies}

Final grades are determined after the entire faculty of the Department of
Computer Science --- not just your course instructor for CMPSC 600 --- review
and discuss all of the submitted deliverables.

Your grade in CMPSC 600 will be based on a combination of the following
activities and deliverables. Percentages are not given because we recognize that
the senior thesis experience differs from one student to the next and that there
are many variables, such as the nature of the project and the availability of
external resources, that can influence the relative importance of these
criteria. However, it is important to note that a large percentage of your grade
depends upon your written thesis proposal, the oral defense of your thesis
proposal, and your two thesis chapters.

\vspace*{-.05in}

\begin{itemize}
  \itemsep -.25em

  \item {\bf Class Participation}: This includes meeting regularly with your first reader. Although the exact details
    about frequency and length of each meeting must be established with your first reader, you should adhere to the
    previously stated schedule. Additionally, this also requires regular contributions, in the form of questions and
    comments, to the course's Slack site.

  \item {\bf Course Repositories}: This involves students creating, at minimum, two version control repositories to
    store (i) their thesis proposal and written chapters and (ii) any relevant source code and data.  Both of your
    readers must have administrative access to your repositories.

    % Additionally, you repositories must be correctly integrated into the appropriate channel in our Slack site. All
    % repositories must have a README.md file that contains full-featured and well-written instructions for creating all
    % of the deliverables under version control.

  \item {\bf Written Proposal}: This document must be approved, in writing by the specified deadline, by the first
    reader for your senior thesis and formatted according to the department's thesis proposal style requirements,
    which is available from the course web site.

  \item {\bf Proposal Defense}: This event is scheduled in consultation with your first and second reader and the
    building coordinator, Pauline Lanzine. Students may not schedule their proposal defense until their thesis proposal
    has been formally approved, in writing, by their first reader; evidence of this approval must be submitted to
    Pauline Lanzine when scheduling the defense.

  \item {\bf Thesis Chapters}: Any two chapters of your final senior thesis must be submitted to the course instructor
    by the aforementioned deadline.  Written in a professional and scientific style, these chapters must be formatted in
    the department's thesis style; please note that this style is available from the course web site and it is different
    from the proposal's style.

\end{itemize}

\vspace*{-.25in}
\subsection*{Details About Course Expectations and Deliverables}
\vspace*{-.15in}

\noindent{\bf Class Participation}: Once your readers have been assigned, you must regularly meet with your first
reader, who will report on your participation when the department's faculty meet to assign final grades.  Students are
expected to come to each meeting with a status update on their progress and a meeting agenda.  Students should conclude
each meeting by listing the tasks that they want to complete before the next meeting. In addition, students should
regularly participate in the discussions on the relevant channels in the Slack site for our course. Your participation
on Slack may involve giving a quick status update to your first reader, inviting your first reader to examine a draft of
your proposal or compile and run a new version of a program, or, within the bounds of the Honor Code, answering a
question from another senior conducting thesis research.  Finally, all students are required to attend and actively
participate in all of the class sessions that will involve both meetings with their research group and informal
discussion during the departmental coffee and tea session. Evidence of regular participation must be submitted to the
course instructor; failure to participate and submit this evidence will result in the reduction of your grade in
CMPSC 600.

\noindent{\bf Course Repositories}: Every student must create at least two version control repositories to store (i)
their thesis proposal and written chapters and (ii) any relevant source code and data; students may create additional
repositories in consultation with their first reader. Unless advised by their first reader to do otherwise, students are
expected to create their repositories in Bitbucket. Regardless of the repository provider that you select, your first
and second reader must have administrative access to your repositories. Additionally, your repositories must be
correctly integrated into the appropriate channel in our Slack site, thereby allowing all faculty and students to see
everyone's progress on their research. Finally, all repositories must have a README.md file that contains well-written
instructions for creating all of the deliverables under version control. Failure to regularly commit to
your repositories will lead to a decrease in your final grade for CMPSC 600.

\noindent{\bf Thesis Proposal}: The proposal should follow the department's proposal style and thus must include an
abstract, the main body of your proposal, a tentative schedule for completing the project, a bibliography, and any other
information deemed important by your first reader. This will often include one or more of the following: a survey of the
existing literature; an overview of your proposed technique; technical diagrams and formal statements of algorithms
illustrating your main approach; the description of an evaluation method; examples or code artifacts or other evidence that
you understand the nature of the work you are proposing and can feasibly complete it in the time available.  Finally, the
proposal must fully adhere to professional standards of writing.

Although your first reader will be your primary contact person as your write and revise your thesis proposal, you may
involve your second reader as appropriate. Primarily, your first reader will make suggestions on your submitted drafts;
students are expected to revise multiple proposal drafts.  You must work at a pace that will ensure that your first
reader can formally approve the final draft of your thesis proposal before the stated deadline. Failure to secure formal
approval of your proposal before this date will result in the reduction of your final grade in CMPSC 600.  Securing
formal approval of your thesis involves a student having their first reader sign and date the final printed version of
the thesis proposal. This document must be shown to Pauline Lanzine when scheduling your proposal defense; no defense
will be scheduled without this evidence of approval.

\noindent{\bf Proposal Defense}: A proposal defense is a prepared, formal presentation of about ten minutes in which you
lay out the essential parts of your chosen project under the assumption that your first and second reader have studied
your proposal.  Following the presentation that is supported by polished slides, you will participate in a discussion
with your readers to identify potential challenges, refine or modify some aspects of the thesis proposal, and ensure that
your project is feasible and appropriate. All components your proposal defense should be completed in consultation with
your first reader.  You must schedule your proposal defense before the stated deadline. Your grade in CMPSC 600 will be
reduced if you miss the deadline for scheduling or conducting your defense.

\noindent{\bf Thesis Chapters}: Your two chapters, due on the previously stated date, should represent a significant
addition to or extension of the material in your proposal. Don't simply ``split the proposal into two chapters'' ---
this usually does not work well since your chapters must represent work completed, not work being proposed. Chapters are
judged according to the same professional standards as the proposal; they must include a full bibliography, a
preliminary table of contents, lists of any figures and tables, and any other items required by your first reader.

As you write your chapters in consultation with your first and second reader, allow these individuals to comment on your
drafts and then make all of their requested changes.  Plan to write several drafts of the chapters before submitting them
on the due date; failure to submit the chapters by the stated deadline will result in the reduction of your final grade
in CMPSC 600.

% \subsubsection*{Seeking Assistance}

% Students who are struggling to understand the knowledge and skills developed in a class or laboratory session are
% encourage to seek assistance from the course instructor. Throughout the semester, students should, within the bounds of
% the Honor Code, ask and answer questions on the Slack site for our course; please request assistance from the instructor
% first through Slack before sending an email. Students who need the course instructor's assistance must schedule a
% meeting through his web site and come to the meeting with all of the details needed to discuss their question.

\vspace*{-.15in}
\subsubsection*{Using Email}
\vspace*{-.05in}

Although we will primarily use Slack for class communication, we will sometimes use email to send announcements about
important class matters. It is your responsibility to check your email at least once a day and to ensure that you can
reliably send and receive emails. This class policy is based on the statement about the use of email that appears in
{\em The Compass}, the student handbook.

\vspace*{-.15in}
\subsection*{Honor Code}
\vspace*{-.05in}

The Academic Honor Program that governs the academic program at Allegheny College is described in the Allegheny
Academic Bulletin.  The Honor Program applies to all work that is submitted for academic credit or to meet non-credit
requirements for graduation at Allegheny College.  This includes all work assigned for these classes (e.g., source code,
technical diagrams, and your written content); deliverables that are nearly identical the work of others will be taken
as evidence of violating the Honor Code. All students who have enrolled in the College will work under the Honor
Program.  Each student who has matriculated at the College has acknowledged the following pledge:

\vspace*{-.1in}
\begin{quote}
I hereby recognize and pledge to fulfill my responsibilities, as defined in the Honor Code, and to maintain the
integrity of both myself and the College community as a whole.
\end{quote}
\vspace*{-.3in}

\subsection*{Disability Services}
\vspace*{-.05in}

The Americans with Disabilities Act (ADA) is a federal anti-discrimination statute that provides comprehensive civil
rights protection for persons with disabilities.  Among other things, this legislation requires all students with
disabilities be guaranteed a learning environment that provides for reasonable accommodation of their disabilities.
Students with disabilities who believe they may need accommodations in this class are encouraged to contact Disability
Services at 332--2898.  Disability Services is part of the Learning Commons and is located in Pelletier Library.
Please do this as soon as possible to ensure that approved accommodations are implemented in a timely fashion.

\vspace*{-.1in}
\subsection*{Welcome to an Adventure in Computer Science}

% Moreover, these courses properly position you to conduct ground-breaking work that can have a positive influence on your
% future career and graduate school prospects, the students and faculty in the Department of Computer Science, the
% Allegheny College community, and a broader society that heavily relies on computer hardware and software.

CMPSC 600 affords you the opportunity to pursue independent research in computer science and to ensure that your work
has a positive influence on your future plans, the students and faculty at Allegheny College, and a broader society that
relies heavily on computer hardware and software.  At the start of your senior year, we invite you to pursue this class
with great enthusiasm and vigor.

% \subsection*{Special Needs and Disabilities}
% The Americans with Disabilities Act (ADA) is a federal anti-discrimination
% statute that provides comprehensive civil rights protection for persons
% with disabilities. Among other things, this legislation requires that
% all students with disabilities be guaranteed a learning environment
% that provides for reasonable accommodation of their disabilities.
% If you believe  you have a disability requiring an accommodation,
% please contact the Learning Commons at 332-2898.
%
\end{document}
